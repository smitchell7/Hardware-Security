The ROP attack was first approached by determining the size and boundaries of the stack (see Figure \ref{fig::stack}). Once the boundaries were determined, we replaced the location which would be popped into the program counter (PC) with the address of Gadget A. Each successive call (shown in Table \ref{tab:gadgets}) was determined by overwriting the values to be placed into the R4 and PC registers.

The attack outlined in Table \ref{tab:gadgets} was performed and resulted in the desired functionality until the device was reset. Upon reset, the program would jump to a scatter function located in the previously erased region. This resulted in an interrupt which prevented the device from executing the program located at the address of main(). This was solved by inserting data falsification at the scatter function. 