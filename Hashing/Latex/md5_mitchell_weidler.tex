\documentclass[conference]{IEEEtran}
\usepackage[ruled,vlined]{algorithm2e}
\usepackage{amsmath}
\usepackage[english]{babel} %localisation
\usepackage{caption,subcaption} %supposedly incompatible with Springer and IOP, IEEETran and ACM SIG
\usepackage{cite} %nice citations, e.g. [1--5]
\usepackage{fixltx2e} %fix latex bugs
\usepackage{graphicx}
\PassOptionsToPackage{hyphens}{url}\usepackage{hyperref} %clickable URLS
\usepackage[htt]{hyphenat} %hyphenate \texttt
\usepackage{microtype} %makes text pretty; also condenses
\usepackage{multirow} %multiple rows in tables
\usepackage{siunitx,textcomp} %\SI{value}{unit}, \si{unit}; textcomp for microtype compatibility
%\usepackage [caption=false]{subfig} %if caption/subcaption not available
% \usepackage{tikz,pgfplots} %drawings and plots
\usepackage[siunitx]{circuitikz} %circuit figures
\usepackage[T1]{fontenc} %ensure proper hyphenation and treatment of math in sentences
\usepackage{booktabs}
\bibliographystyle{IEEEtran}

\begin{document}

% paper title
% can use linebreaks \\ within to get better formatting as desired
\title{Efficient execution of the MD5 algorithm in password recovery applications}

% author names and affiliations
% use a multiple column layout for up to three different
% affiliations
\author{\IEEEauthorblockN{Sam Mitchell and Nathanael Weidler}
\IEEEauthorblockA{Deptartment of Electrical and Computer Engineering\\
Utah Stat University\\
Logan, Utah 84322\\
e-mail: samuel.alan.mitchell@gmail.com, NWeidler@gmail.com}
}

% make the title area
\maketitle


\begin{abstract}
%\boldmath
% Summarize project and results (executive summary).
This paper describes the implementation of high-throughput password cracking devices. We consider architecture-aware implementations of password crackers on FPGA and X86 architectures. An analysis of speed and cost efficacy is included. 

% This paper considers the efficient computation of passwords. Multiple methods to increase hashing throughput are presented. It is shown that hardware implementation of a password cracker provides more throughput per unit dollar than a software implementation. Future research will address the efficacy of different architectures in password computation. 
\end{abstract}

\begin{IEEEkeywords}
Passwords, MD5, authentication, password-cracking algorithms, data security.
\end{IEEEkeywords}

\section{Introduction}
	Hashing is one method of securely storing passwords or verifying the validity of data. Efficient methods of analyzing large numbers of hashes is necessary for password recovery.  

	MD5 is a computationally simple operation with minimal execution time. An inefficient implementation of MD5 is sufficient for everyday hashing needs, however password recovery applications require an efficient method to test large numbers of passwords. 

	This paper discusses two high-throughput architecture-informed implementations of the MD5 algorithm. 

% Describe the problem and what you're doing to address it (i.e.\ building architecture-informed md5 password cracker for software and hardware; how password cracking operates).  

% \subsection{Related work}
% 	Previous work in efficient implementations of 
% What is needed to understand your optimisations and/or implementations.  Do you draw on existing techniques? If so, describe and cite them.

\subsection{Structure of paper}
	The organization is as follows: in Section II, the development of a software-based MD5 password cracking device is presented and analyzed. Section III contains the focus and development of a hardware-based MD5 password cracking device. Conclusions are presented in Section IV. 

\section{Software implementation}
\section{Software}
\subsection{Optimizations}

\section{Hardware implementation}
\section{Hardware}

hello

%

% Table generated by Excel2LaTeX from sheet 'Sheet1'


\section{Conclusion}
% Summarize results and lessons learned.
	This paper considers the efficient computation of passwords. Multiple methods to increase hashing throughput are presented. It is shown that hardware implementation of a password cracker provides more throughput per unit dollar than a software implementation. Future research will address the efficacy of different architectures in password computation. 
\bibliography{report}
% that's all folks
\end{document}



\end{document}